\documentclass{article}
%%%%%%%%%%%%%%%%%%%%%%%%%
%%% PACKAGE INCLUSION %%%
%%%%%%%%%%%%%%%%%%%%%%%%%

%% AMS packages for special symbols, etc
\usepackage{amsmath,amssymb}
\usepackage{calligra}

%% This package gives you coloured text and various other simple
%% graphics hacks.  For details, see documentation in 
%% in /usr/local/teTeX/texmf/doc/generic/pstricks/*
\usepackage{pstricks}
\newrgbcolor{darkblue}{0.1 0.1 0.5}

%% The textpos package is necessary to position textblocks at arbitary 
%% places on the page.  Use showboxes option to show outlines of textboxes.
\usepackage[absolute]{textpos}
%\usepackage[absolute,showboxes]{textpos}

%% Package to include graphics.  
\usepackage{graphicx}

% Color box
\usepackage{tcolorbox}
\usepackage{multicol}
\usepackage{wrapfig}
\usepackage[framemethod=TikZ]{mdframed}

%% Define path for figures -- for safety, keep the last /
\graphicspath{{/your/figure/directory/here/if/any/}
{/an/second/directory/path/can/go/here/}}

%% Wrap text around figures
%\usepackage{wrapfig}

%% Use Times font instead of Computer Modern -- this gives better
%% appearance when resizing to large sixes.
%% Note that without the ``G0'' in the dvips conversion, 
%% all character combinations that will normally result in 
%% ligatures will have to be hacked to display properly.  For example, 
%%     fi --> \mbox{f}\mbox{i}
%% Other characters may also fail.  In addition, the Mathtimes font 
%% set should really be used for mathematics, but unfortunately they 
%% are only proprietary.  (The Computer Modern fonts may still look OK.)
\usepackage{times}

%% These colors are tried and tested for titles and headers. Don't
%% over use color!
%\usepackage[usenames]{color} % commented by Karol Kozioł
\usepackage{xcolor}
\definecolor{DarkBlue}{rgb}{0.1,0.1,0.5}
\definecolor{Black}{rgb}{0.0,0.0,0.0}
\definecolor{Red}{rgb}{0.9,0.0,0.1}
\definecolor{DarkBlue2}{rgb}{0.00,0.08,0.6}
\definecolor{DarkRed2}{rgb}{0.6,0.00,0.08}
\definecolor{DarkGreen2}{rgb}{0.00,0.6,0.08}

%% Load shadow box package
%\usepackage{shadow}

%% This loads font sizes in style file a0size
\usepackage{a0size}


%%%%%%%%%%%%%%%%%%%%%%%%%%%%%%%% 
%%% NEW COMMAND DEFINTITIONS %%%
%%%%%%%%%%%%%%%%%%%%%%%%%%%%%%%%

%% See documentation for a0poster class for the size options here
%%    \normalsize will produce smaller type that might look too small
%%    \large will produce larger type
%% Feel free to modify if you want a different look
%\let\Textsize\normalsize
\let\Textsize\large
\def\RHead#1{\noindent\hbox to \hsize{\hfil{\LARGE\color{DarkBlue} #1}}\bigskip}
\def\LHead#1{\noindent{\LARGE\color{DarkBlue} #1}\bigskip}
\def\CHead#1{\begin{center}\noindent{\LARGE\color{DarkBlue} #1}\end{center}}
\def\redhead#1{\begin{center}\noindent{\LARGE\color{red} #1}\end{center}}
\def\orangehead#1{\begin{center}\noindent{\LARGE\color{orange} #1}\end{center}}
\def\greenhead#1{\begin{center}\noindent{\LARGE\color{green} #1}\end{center}}
\def\Subhead#1{\noindent{\Large\color{black} #1}\bigskip}
\def\Title#1{\noindent{\textbf{\Huge\color{black} #1}}}

\def\big#1{{\Large{\textbf{#1}}}}


%%%%%%%%%%%%%%%%%%%%%%%%%%%%%
%%% GLOBAL LAYOUT OPTIONS %%%
%%%   NUMBER OF COLUMNS   %%%
%%%%%%%%%%%%%%%%%%%%%%%%%%%%%

%% Set paper size
%% Depending on the conference, the posterboard size may be different.
%% This template was based on an ISO standard A0, which is in use everywhere
%% except for the United States.  A0 paper is  46.81 in x 33.11 in.
%% Depending on the posterboard size and the printer, you may need to 
%% change the widths and margins here.  Text width and height are set
%% in terms of paper width and height -- you can change margins here.
\setlength{\paperwidth}{32in}
\setlength{\paperheight}{40in}
\setlength{\textwidth}{26in}    %% paperwidth - (3in)
\setlength{\textheight}{36in}   %% paperheight - (3in)
\special{papersize=\the\paperwidth,\the\paperheight}
\typeout{Paper width and height are \the\paperwidth and \the\paperheight}
\typeout{Text width and height are \the\textwidth and \the\textheight}
%% Margins
\setlength{\headheight}{0cm}
\setlength{\headsep}{0cm}
\setlength{\topmargin}{1in}
\setlength{\topskip}{0cm}
\setlength{\oddsidemargin}{1in}
\setlength{\evensidemargin}{0in}
%% Font sizes
\renewcommand{\tiny}{\fontsize{12}{14}\selectfont}
\renewcommand{\scriptsize}{\fontsize{14.4}{18}\selectfont}   
\renewcommand{\footnotesize}{\fontsize{17.28}{22}\selectfont}
\renewcommand{\small}{\fontsize{20.74}{25}\selectfont}
\renewcommand{\normalsize}{\fontsize{24.88}{30}\selectfont}
\renewcommand{\large}{\fontsize{29.86}{37}\selectfont}
\renewcommand{\Large}{\fontsize{35.83}{45}\selectfont}
\renewcommand{\LARGE}{\fontsize{43}{54}\selectfont}
\renewcommand{\huge}{\fontsize{51.6}{64}\selectfont}
\renewcommand{\Huge}{\fontsize{61.92}{77}\selectfont}
\newcommand{\veryHuge}{\fontsize{74.3}{93}\selectfont}
\newcommand{\VeryHuge}{\fontsize{89.16}{112}\selectfont}
\newcommand{\VERYHuge}{\fontsize{107}{134}\selectfont}
%% skip lengths
\setlength{\smallskipamount}{6pt plus 2pt minus 2pt}
\setlength{\medskipamount}{12pt plus 4pt minus 4pt}
\setlength{\bigskipamount}{24pt plus 8pt minus 8pt}
\setlength{\abovecaptionskip}{25pt}
\setlength{\belowcaptionskip}{0pt}
\setlength{\abovedisplayskip}{25pt plus 6pt minus 15 pt}
\setlength{\abovedisplayshortskip}{0pt plus 6pt}
\setlength{\belowdisplayshortskip}{13pt plus 7pt minus 6pt}
\setlength{\belowdisplayskip}{\abovedisplayskip}

%% Set up the grid
%%
%% Note that [40mm,40mm] is the margin round the edge of the page
%% it is _not_ the grid size. That is always defined as 
%% PAGE_WIDTH/HGRID and PAGE_HEIGHT/VGRID. In this case we use
%% 46 x 26. This gives us 4 columns of width 10 boxes, with a gap of
%% width 2 in between them.  There are 26 vertical boxes.
%%
%% (Note however that texblocks can be positioned fractionally as well,
%% so really any convenient grid size can be used.)
%%
\TPGrid[40mm,40mm]{26}{46}      % 4 cols of width 10, plus 3 gaps width 2

%% Text layout parameters
\parindent=0pt
\parskip=0.5\baselineskip





%%%%%%%%%%%%%%%%%%%%%
%%% THEOREMS, ETC %%%
%%%%%%%%%%%%%%%%%%%%%
\newtheorem{thm}{Theorem}






%%%%%%%%%%%%%%%%%%%%%%%%%%%%
%%% DOCUMENT BEGINS HERE %%%
%%%%%%%%%%%%%%%%%%%%%%%%%%%%

%% The basic format of the poster is to create text boxes with the
%% various things you want to display.  You can then play around 
%% with how to lay thing out.  In the old version of this template,
%% the content was always provided with alternatives suitable for
%% printing on sheets of paper (resizing fonts, images, etc).  I
%% think that's too confusing to read.  The old layout was:
%%     \ifposter
%%          some poster commands go here
%%     \else
%%          an alternative style here in case you are printing on
%%          regular sheets of paper
%%     \fi
%% One option is to make the entire poster and then wrap it in an \ifposter
%% and then make all the slides separately.  This seems to be easier
%% if your poster is not much like a bunch of 8.5x11 sheet tacked together
%% in the first place.
\begin{document}

%% Do not put page numbers at the bottom of the page for poster
\pagestyle{empty}



%% Declare proper hyphenation
\hyphenation{equi-bi-ax-i-al}
\hyphenation{in-fin-i-tes-i-mal}


%% Border and background options -- you can make up others if you
%% like.  These all use the pstricks package.
%% DRAW A BLUE BORDER AROUND THE POSTER USING PSTRICKS
\psset{linewidth=0.5cm}
% Sets up lengths for frame
\newlength{\frameleft}
\newlength{\frameright}
\newlength{\frametop}
\newlength{\framebottom}
\setlength{\frameleft}{-1in}
\setlength{\frameright}{\textwidth}
\addtolength{\frameright}{1in}
\setlength{\frametop}{1in}
\setlength{\framebottom}{-\textheight}
\addtolength{\framebottom}{-1in}
% Draws a blue frame
%\psframe[linecolor=darkblue,cornersize=absolute,linearc=2]
%(\frameleft,\framebottom)(\frameright,\frametop)% need to overlay EOSMLS
%\psline{->}(0cm,0cm)(\textwidth,-\textheight)
%   *** End code to draw border *** 

%% USE A COLORED BACKGROUND FOR THE ENTIRE POSTER
%% [ADS 4-2005] THIS OPTION IS NOT SUPPORTED YET
%\newrgbcolor{gradbegin}{0.3 0.5 0.7}
%\newrgbcolor{LightBlue}{0.7 0.7 1.0}
%\psframe[fillstyle=solid,fillcolor=LightBlue](\frameleft,\framebottom)(\frameright,\frametop)



%% Adjust spacing in long displayed mathematical formulas to tighten them up
\setlength{\abovedisplayskip}{0.75\abovedisplayskip}
\setlength{\belowdisplayskip}{0.75\belowdisplayskip}

%%% General font size
\large{}

%### SEPARATOR
%%\begin{textblock}{32}(8,5)
%%	\includegraphics[width=10in]{imgs/divider.eps}
%%\end{textblock}

%% Understanding textblocks is the key to being able to do a poster in
%% LaTeX. The first argument gives the block width in grid cells, the
%% second gives the positioning on the grid.
%%
%% NOTE:  You will have to do a lot of previewing to get everything
%% in the right place.
%%
\begin{textblock}{26}(00,00)
	%\begin{tcolorbox}[boxsep=12pt, colback=gray!10!white]
	\begin{tcolorbox}[boxsep=12pt]%, colback=white]
		\begin{center}
			\Title{LEADER: Low-Rate Denial-of-Service Attacks Defense} \\
			\vspace{.5in}
			\LHead{Rajat Tandon - Haoda Wang - Nicolaas Weideman - Christophe Hauser - Jelena Mirkovic}\\ 
			\Subhead{\textit{Information Sciences Institute, University of Southern California}}
			%\vspace{.2in}
		\end{center}

	\end{tcolorbox}
\end{textblock}

%\begin{textblock}{46}(00,01.5)
%\begin{center}
%\LHead{Christophe Hauser - Yan Shoshistaishvili - Ruoyu Wang}\\
%\LHead{\textit{ {\large{University of Southern California - Arizona State University - University of California, Santa Barbara}} }}
%\end{center}
%\end{textblock}


%% UCB EECS logo on left, Wireless Foundations Logo on right
\begin{textblock}{8}(0,0)
	\begin{center}
		\vspace{.15in}
		%\begin{tcolorbox}[width=.46\textwidth, colback=black!100!white]
			%\includegraphics[height=7cm]{chris.jpg}
		%\end{tcolorbox}
			%\color{white}{hauserc@usc.edu}
%\end{tcolorbox}
%\colorbox{darkgray}{\color{white}{\par hauserc@usc.edu \par}}
		\end{center}
	\end{textblock}

\begin{textblock}{8}(38,01)
\begin{center}
%\includegraphics[height=5cm]{wifound.eps} % modified by Karol Kozioł
%\includegraphics[height=2.5cm]{wifound.eps}
\end{center}
\end{textblock}


\begin{textblock}{42}(02,03)
\begin{center}
%\rule{1200pt}{10pt}
\end{center}
\end{textblock}

\begin{textblock}{8}(11,06)
	{\Huge{\calligra Overview}}
\end{textblock}
%% Begin 1st row
\begin{textblock}{8}(00,06)
%\CHead{Introduction}            %% \CHead creates a centered title
	\begin{tcolorbox}[boxsep=12pt,colback=yellow!10!white]
Low-rate denial-of-service(LRD) attacks are often hard to detect at the network level as they consume little bandwidth. It is the intricacies in the payloads and the dynamics of the attack traffic that induces denial-of-service on servers when processed by specific hardware and software.

We introduce \textbf{Leader, a hybrid approach for application-agnostic and attack-agnostic detection and mitigation of LRD attacks.} Leader operates by \textbf{learning normal patterns of network, application and system-level resources} when processing legitimate external requests. It relies on a \textbf{novel combination of runtime, system and network monitoring and offline binary program analysis.}
\end{tcolorbox}
\end{textblock}


%%\begin{textblock}{1}(00,13)
%%	\includegraphics[width=\textwidth]{imgs/why_binary.eps}
%%\end{textblock}

\begin{textblock}{6}(0,16)
%\CHead{Why binary analysis ?}
\begin{figure*} 
	\begin{center}
		\includegraphics[width=1.2\textwidth]{Overview-a.png}
			\caption{Monitoring at different abstraction levels, with the associated trade-off.  \label{fig:monitor}}
	\end{center}
\end{figure*}
\end{textblock}

\begin{textblock}{6}(8,16)
\begin{figure*} 
	\begin{center}
	   \includegraphics[width=1.5\textwidth]{Overview-b.png}
	   	\caption{System overview.  \label{fig:overview}}
	\end{center}
\end{figure*}
\end{textblock}

\begin{textblock}{9}(8.5,8)
Monitoring and system overview: \textbf{Novelty of our approach lies in our connection life stages and code path abstractions, which are built from monitoring the system at network, OS and application levels.} Figure \ref{fig:monitor} shows a high-level view of the abstraction levels at which Leader operates. It also illustrates the trade-offs between accuracy of semantic reasoning and, monitoring cost and delay. OS-level instrumentation allows an observer to gain more insights about the application’s semantics and program analysis offers the highest level of semantic reasoning. Figure \ref{fig:overview} gives the system overview. LRD attacks usually involve several incoming service requests arriving at the server that are expensive/slow processing leading to resource depletion.
%	\begin{center}
%		\includegraphics[width=1.3\textwidth]{LEADER-Architecture}
%	\end{center}

%\end{multicols}
\end{textblock}


\begin{textblock}{8}(18,18)
	\begin{tcolorbox}[boxsep=12pt, colback=red!10!white, colframe=red!80!white]
\redhead{Preliminary Results}
%Analyzing programs at the binary level involves some \textbf{challenges}. Compared to source-level analysis, the amount of \textbf{semantic information} available from the disassembly of binary code is \textbf{limited}. The following describes some of these challenges.

We relied on on Emulab to experiment with Slowloris, a common type of low-rate denial-of-service attack. We set up a static Web server, and used 10 legitimate and one attack client. The legitimate clients continuously requested the main Web page, using wget, each 200 ms. This created 50 requests per second at the server. The attack client opened 1,000 simultaneous attack connections and kept them open for as long as possible by sending never-ending headers on each. This had negative impact on legitimate clients, that had trouble establishing connection with the server. Figure \ref{fig:slowloris} shows the life stage diagram for traffic in attack case, which includes both legitimate
and attack connections, and compared it to the diagram in the baseline case. The highlighted stages differ between two cases and help us identify anomalous connections.
	\end{tcolorbox}
\end{textblock}

\begin{textblock}{17.5}(0,26)
\begin{tcolorbox}[boxsep=10pt,colback=green!1!white]

Another novel aspect of Leader lies in the structures it uses to capture sequences of resource-use events in a temporal and relational manner per each incoming service request. These sequences are known as \textbf{connection life stages}. 
%%They are built from multiple, complementary observations collected at the (1) network level, (2) OS level and (3) application level. The connection life stages are then clustered into typical resource-use patterns, or profiles for applications and for users. 
We relied on on Emulab to experiment with Slowloris, a common type of low-rate denial-of-service attack. We set up a static Web server, and used 10 legitimate and one attack client. The legitimate clients continuously requested the main Web page, using wget, each 200 ms. This created 50 requests per second at the server. The attack client opened 1,000 simultaneous attack connections and kept them open for as long as possible by sending never-ending headers on each. This had negative impact on legitimate clients, that had trouble establishing connection with the server. Figure \ref{fig:slowloris} shows the life stage diagram for traffic in attack case, which includes both legitimate
and attack connections, and compares it to the diagram in the baseline case.
%The highlighted stages differ between two cases and help us identify anomalous connections.
%We use these profiles to detect anomalous use and characterize LRD attacks. We also design mitigation actions that remove attack traffic, or increase system’s robustness to the specific attack with the aid of these profiles. 
%It captures the connection, client, application and whole-device profiles at all times. During normal operation, it learns the legitimate behavior of the connections, applications, clients and the device where it is deployed and signals an attack by comparing the  instantaneous profiles to corresponding baseline profiles and identifies troubled services.
\end{tcolorbox}
\end{textblock}

%### SEPARATOR ###
\begin{textblock}{26}(0,32.5)
\begin{tcolorbox}[boxsep=10pt,colback=yellow!1!white]
\begin{figure*} 
	\includegraphics[width=27in]{Slowloris}
	\caption{Life-stage diagrams for Slowloris attack: highlighted items show anomalies.  \label{fig:slowloris}}
\end{figure*}
\end{tcolorbox}
\end{textblock}





%%\begin{textblock}{26}(0,34)
%%\begin{tcolorbox}[boxsep=1pt,colback=yellow!5!white]
%%	\includegraphics[width=27in]{Hash-Collision}
%%\end{tcolorbox}
%%\end{textblock}

\begin{textblock}{26}(0,42)
\begin{tcolorbox}[boxsep=1pt,colback=yellow!5!white]
	We are working on \textbf{Systemtap} to build aggregate profiles of an application's/system's resource usage pattern for each connection at the system call level. We capture the time spent on each system call and the resources used by them such as memory usage, number of page faults and open file descriptors, cpu cycles and other thread/process level details. We do such profiling for both legitimate traffic as well as attack traffic. There are notable differences between the two. We will utilize these profiles, along with other sophistication, to detect attacks and mitigate them. 
\end{tcolorbox}
\end{textblock}
%%\includegraphics[width=10in]{Slowloris}


%##################### Second line #################



%########################### THIRD LINE ####################################

%%\begin{textblock}{8}(11,34)
%%	{\Huge{\calligra Next Steps}}
%%\end{textblock}

%### Analysis scope VS accuracy
%%\begin{textblock}{8}(0,36)
%%	\begin{center}
%%		\CHead{Analysis scope vs accuracy}
%%	\end{center}

%%	\begin{wrapfigure}{l}{0pt}
%%		\includegraphics[width=.17\textwidth]{imgs/diff.eps}
%%	\end{wrapfigure}
%\emph{Closing the gap} between code size/complexity and the level of accuracy that can be achieved with automated tools is a research direction of interest.
%%	Leverage time or space locality in code changes (\textit{e.g.} patches) to \textbf{infer semantic information} and specialize the analyses: \textit{Incremental vulnerability discovery}. \\ 

%%	\begin{tcolorbox}[boxsep=12pt, colback=green!10!white, width=\textwidth]%, colframe=darkblue, title=Analysis scope vs accuracy]
%%		Derive \textbf{new techniques and heuristics} to statically detect dangerous code changes in complex code bases.
%%	\end{tcolorbox}
%%\end{textblock}

%### Reducing semantic gaps ###
%%\begin{textblock}{8}(9,35)
%%	\begin{center}
%%		\CHead{Towards reducing semantic gaps}
%%	\end{center}

%%	\begin{wrapfigure}{l}{0pt}
%%		\includegraphics[width=.17\textwidth]{imgs/similarity.eps}
%%	\end{wrapfigure}

%%By leveraging the large language and architecture support offered by \textbf{LLVM} under the same \textbf{intermediate representation}, and by expanding on the success of \textbf{code similarity techniques} to identify semantic bug patterns ``at scale''.

%%\end{textblock}
%%\begin{textblock}{8}(9,41)
%%	\begin{tcolorbox}[boxsep=12pt, colback=blue!10!white, width=\textwidth] %colframe=darkblue, title=Reducing semantic gaps]
%%		\textbf{Learning semantic constructs} from source/applying ``in-the-wild'' to binaries.
%%	\end{tcolorbox}
%%\end{textblock}


%### IMPROVING MODELS ###

%### MORE INFO ###
\begin{textblock}{8}(18,6)
%\begin{tcolorbox}[title=More information, boxsep=12pt, colframe=orange, colback=white]
\begin{tcolorbox}[boxsep=6pt, colback=orange!30!white]
	\orangehead{\textbf{\\Methodology\\\\}}
	
	\begin{itemize}
		\item \textbf{Behavior profiling}
\\
\\LEADER relies on the collection of measures and statistics of system resources’ usage for successful attack detection. These measures are collected per each incoming service request and comprise observations at the network, system and application level. Leader captures the connection, client, application and whole-device profiles at all times. During normal operation, Leader performs profiling to learn the legitimate behavior of the connections, applications, clients and the device where it is deployed.\\
		\item \textbf{ Attack detection (and characterization)}
		\\
		\\Another novel aspect of leader is that it uses a hybrid detection approach combining network security mechanisms with OS and program-level aspects. Our attack detection module compares instantaneous profiles to corresponding baseline profiles, with the goal of detecting evidence of troubled services that have low instantaneous percentages of successfully served incoming requests, compared to the historical (baseline) percentages of successfully served request.\\
		\item \textbf{Attack mitigation}
		\\
		\\Leader deploys a combination of attack mitigation approaches that includes (1) Derivation of the attack signature from attack connections (2) Costly connection termination, (3) Blacklisting of attack sources, (4) Dynamic resource
replication, (5) Program patching and algorithm modification, and (6) Blacklisting of sources with anomalous profiles.\\
	\end{itemize}

%%	Leader uses Connection Life Stages, a special structure to capture sequences of resource use events in a temporal and relational manner per each incoming service request. It captures the connection, client, application and whole-device profiles at all times. During normal operation, it learns the legitimate behavior of the connections, applications, clients and the device where it is deployed and signals an attack by comparing the  instantaneous profiles to corresponding baseline profiles and identifies troubled services.
	\end{tcolorbox}
\end{textblock}

%### BOTTOM LINE ####
\begin{textblock}{26}(0,45)
	\begin{tcolorbox}[boxsep=12pt, colback=black!90!white, width=\textwidth]%, height=.3in]
		\centering
		\color{white}{Contact us at |} 
		%\hspace{.2in}
		\color{white}{tandon, haodawa, nweidema, hauser, sunshine @isi.edu}
	\end{tcolorbox}
\end{textblock}
\end{document}
